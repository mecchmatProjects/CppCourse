\documentclass[]{article}
\usepackage{lmodern}
\usepackage{amssymb,amsmath}
\usepackage{ifxetex,ifluatex}


\usepackage[utf8]{inputenc}
\usepackage[english,russian,ukrainian]{babel}

\usepackage{fixltx2e} % provides \textsubscript
\ifnum 0\ifxetex 1\fi\ifluatex 1\fi=0 % if pdftex
  \usepackage[T1]{fontenc}
  \usepackage[utf8]{inputenc}
\else % if luatex or xelatex
  \ifxetex
    \usepackage{mathspec}
  \else
    \usepackage{fontspec}
  \fi
  \defaultfontfeatures{Ligatures=TeX,Scale=MatchLowercase}
\fi
% use upquote if available, for straight quotes in verbatim environments
\IfFileExists{upquote.sty}{\usepackage{upquote}}{}
% use microtype if available
\IfFileExists{microtype.sty}{%
\usepackage{microtype}
\UseMicrotypeSet[protrusion]{basicmath} % disable protrusion for tt fonts
}{}
\usepackage[unicode=true]{hyperref}
\hypersetup{
            pdfborder={0 0 0},
            breaklinks=true}
\urlstyle{same}  % don't use monospace font for urls
\usepackage{graphicx,grffile}
\makeatletter
\def\maxwidth{\ifdim\Gin@nat@width>\linewidth\linewidth\else\Gin@nat@width\fi}
\def\maxheight{\ifdim\Gin@nat@height>\textheight\textheight\else\Gin@nat@height\fi}
\makeatother
% Scale images if necessary, so that they will not overflow the page
% margins by default, and it is still possible to overwrite the defaults
% using explicit options in \includegraphics[width, height, ...]{}
\setkeys{Gin}{width=\maxwidth,height=\maxheight,keepaspectratio}
\IfFileExists{parskip.sty}{%
\usepackage{parskip}
}{% else
\setlength{\parindent}{0pt}
\setlength{\parskip}{6pt plus 2pt minus 1pt}
}
\setlength{\emergencystretch}{3em}  % prevent overfull lines
\providecommand{\tightlist}{%
  \setlength{\itemsep}{0pt}\setlength{\parskip}{0pt}}
\setcounter{secnumdepth}{0}
% Redefines (sub)paragraphs to behave more like sections
\ifx\paragraph\undefined\else
\let\oldparagraph\paragraph
\renewcommand{\paragraph}[1]{\oldparagraph{#1}\mbox{}}
\fi
\ifx\subparagraph\undefined\else
\let\oldsubparagraph\subparagraph
\renewcommand{\subparagraph}[1]{\oldsubparagraph{#1}\mbox{}}
\fi

\date{}


\usepackage{enumitem}
\makeatletter
\newcommand{\xslalph}[1]{\expandafter\@xslalph\csname c@#1\endcsname}
\newcommand{\@xslalph}[1]{%
    \ifcase#1\or а\or б\or в\or г\or д\or e\or є\or ж\or з\or i%
    \or й\or к\or л\or м\or н\or о\or п\or р\or с\or т%
    \or у\or ф\or х\or ц\or ч\or ш\or ю\or я\or аа\or бб\or вв%
    \else\@ctrerr\fi%
}
\AddEnumerateCounter{\xslalph}{\@xslalph}{m}
\makeatother


\begin{document}


\newpage
\subsection{13. Робота з класом рядок на С++.}
\setcounter{subsection}{1}


\begin{itemize}

\item
  Які конструктори для класу рядок? Які копі-конструкторів? Скільки та
  які оператори є перевантаженими для класу рядок?
\item
  Як видалити підрядок використовуючи методи класу String?
\item
  Як можна проітеруватись по рядку Сі++?
\item
  Як узнати довжину рядка?
\item
  Як знайти перше входження даного підрядку в рядку? Останнє?
\item
  Як вивести всі слова в реченні, що розділено пробілами? Комами?
\end{itemize}

Задачі для аудиторної роботи

В даній групі задач потрібно реалізувати функції та в тих функціях де
потрібно виводити рядок зробіть 2 варіанти: 1) Результат записати в
новий рядок. 2) Результат замінює рядок, що є аргументом функції.

\begin{enumerate}
\def\labelenumi{\arabic{enumi})}
\item
  Даний рядок, що складається з символів латинського алфавіту,
  розділених пробілами (одним або декількома). Визначити кількість слів,
  які починаються і закінчуються однією і тією ж літерою.
\item
  Даний рядок, що складається з символів латинського алфавіту,
  розділених пробілами (одним або декількома). Перетворити кожне слово в
  рядку, видаливши з нього всі входження останньої літери цього самого
  слова (кількість пропусків між словами не змінювати).
\item
  Даний рядок із символів латинського алфавіту. Перевірте правильність
  розстановки тега \textless{}td\textgreater{}: кожному відкритого тегу
  повинен відповідати закритий \textless{}/ td\textgreater{}.
\item
  Даний рядок -- речення з символів латинського алфавіту. Вивести
  найкоротше слово в реченні. Якщо таких слів декілька, то: 
  а) вивести перше з них; б) останнє з них в) всі такі слова.
\item
  У реченні, що складається зі слів, відокремлених одним пропуском,
  замінити першу букву у слів, що настають за словами die, der, das, на
  прописну.
\item
  Напишіть функцію часткового спліттінгу рядку, тобто функцію, що
  приймає рядок та повертає перше слово з рядку (роздільник -- задається
  як аргумент функції)
\item
  Напишіть функцію, що приймає рядок та повертає масив (як
  аргумент-змінний) всі дійсні числа, що містяться в рядку (роздільник
  -- задається як аргумент функції)
\item
  У записці слова зашифровані - кожне з них записано навпаки.
  Розшифрувати повідомлення
\end{enumerate}

Задачі для самостійної роботи

\begin{enumerate}
\def\labelenumi{\arabic{enumi})}
\setcounter{enumi}{8}
\item
  Даний рядок, що складається з символів латинського алфавіту,
  розділених пробілами (одним або декількома). Вивести рядок, що містить
  ці ж слова, але розділені одним символом ',' (кома). В кінці
  поставити крапку.
\item
  Даний рядок, що складається з символів латинського алфавіту,
  розділених пробілами (одним або декількома). Перетворити кожне слово в
  рядку, видаливши з нього всі входження останньої літери цього слова
  (кількість пропусків між словами не змінювати).
\item
  Речення складається з слів, розділених одним або декількома
  пропусками або комами. Написати програму, що друкує все слова, що закінчуються на
  заданий символ.
\item
  Даний рядок, що складається з символів латинського алфавіту,
  розділених пробілами (одним або декількома). Перетворити кожне слово в
  рядку видаливши з нього всі входження заданого символу (кількість
  пропусків між словами не змінювати).
\item
  Даний рядок-речення з символів латинського алфавіту. Перетворити рядок
  так, щоб кожне слово починалося з великої літери.
\item
  Даний рядок-речення з символів латинського алфавіту. Вивести найдовше
  слово в реченні (якщо таких слів кілька, то вивести останнє з них).
\item
  Визначити, скільки разів в рядку зустрічається задане слово.
\item
  У записці слова зашифровані -- кожне з них записано навпаки. Розшифрувати повідомлення.
\item
  Даний рядок, що складається з символів латинського алфавіту,
  розділених пробілами (одним або декількома). Визначити кількість слів,
  які містять введений символ.
\item
  Речення складається з слів, розділених одним або декількома
  пропусками. Написати програму, що друкує все слова, що закінчуються на
  введений символ.
\item
  У англійському реченні слова розділені одним пропуском. У всіх словах,
  наступних за артиклями a, an та the, першу букву замінити на прописну.
  Написати програму, що виконує цю роботу.
\item
  Написати програму, що визначає, який відсоток слів в англійському
  тексті містить подвоєну приголосну.
\item
  У мові використовується латинський алфавіт, причастя завжди
  закінчується суфіксом "ings". Задана рядок слів, в якій слова
  відокремлюються одним або декількома пропусками. Надрукувати причастя
  з цього рядку.
\item
  Даний рядок з малих символів латинського алфавіту. Замініть кожен
  символ на наступний за ним за алфавітом, символ 'z' замініть на 'a'.
\item
  Даний рядок із символів латинського алфавіту. Замініть всі входження
  рядків ``one'', "two","three",\ldots{},''nine'' на символи `1',
  '2','3',\ldots{},'9'.
\item
  Відредагувати задане речення, видаляючи з нього ті слова, які
  зустрічаються в реченні задану кількість разів.
\item
  Визначте, який відсоток символи кожного слова складають з символів
  даного речення.
\item
  Дан текст, що складається з символів латинського алфавіту, пробілів і
  знаків пунктуації. Знайдіть найпоширенішу голосну букву (без
  урахування регістру).

\end{enumerate}

Додаткові задачі:

\begin{enumerate}
\def\labelenumi{\arabic{enumi})}
\setcounter{enumi}{27}
\item
  Даний рядок в якому зустрічаються слова, які складаються з восьми
  цифрових символів. Переведіть всі їх в формат дати "dd-mm-yyyy" і
  перевірте коректність такої дати.
\item
  В текстовому файлі записані в кожному рядку значення поліномів за
  допомогою знаків +, -, *, **(ступінь) та цифр і літери $x$. Введіть
  значення $x$ з консолі та для всіх коректних записів поліномів обчисліть
  їх значення для даного $x$ та виведіть в новий текстовий файл.
\end{enumerate}

\end{document}
