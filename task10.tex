\documentclass[]{article}
\usepackage{lmodern}
\usepackage{amssymb,amsmath}
\usepackage{ifxetex,ifluatex}


\usepackage[utf8]{inputenc}
\usepackage[english,russian,ukrainian]{babel}

\usepackage{fixltx2e} % provides \textsubscript
\ifnum 0\ifxetex 1\fi\ifluatex 1\fi=0 % if pdftex
  \usepackage[T1]{fontenc}
  \usepackage[utf8]{inputenc}
\else % if luatex or xelatex
  \ifxetex
    \usepackage{mathspec}
  \else
    \usepackage{fontspec}
  \fi
  \defaultfontfeatures{Ligatures=TeX,Scale=MatchLowercase}
\fi
% use upquote if available, for straight quotes in verbatim environments
\IfFileExists{upquote.sty}{\usepackage{upquote}}{}
% use microtype if available
\IfFileExists{microtype.sty}{%
\usepackage{microtype}
\UseMicrotypeSet[protrusion]{basicmath} % disable protrusion for tt fonts
}{}
\usepackage[unicode=true]{hyperref}
\hypersetup{
            pdfborder={0 0 0},
            breaklinks=true}
\urlstyle{same}  % don't use monospace font for urls
\usepackage{graphicx,grffile}
\makeatletter
\def\maxwidth{\ifdim\Gin@nat@width>\linewidth\linewidth\else\Gin@nat@width\fi}
\def\maxheight{\ifdim\Gin@nat@height>\textheight\textheight\else\Gin@nat@height\fi}
\makeatother
% Scale images if necessary, so that they will not overflow the page
% margins by default, and it is still possible to overwrite the defaults
% using explicit options in \includegraphics[width, height, ...]{}
\setkeys{Gin}{width=\maxwidth,height=\maxheight,keepaspectratio}
\IfFileExists{parskip.sty}{%
\usepackage{parskip}
}{% else
\setlength{\parindent}{0pt}
\setlength{\parskip}{6pt plus 2pt minus 1pt}
}
\setlength{\emergencystretch}{3em}  % prevent overfull lines
\providecommand{\tightlist}{%
  \setlength{\itemsep}{0pt}\setlength{\parskip}{0pt}}
\setcounter{secnumdepth}{0}
% Redefines (sub)paragraphs to behave more like sections
\ifx\paragraph\undefined\else
\let\oldparagraph\paragraph
\renewcommand{\paragraph}[1]{\oldparagraph{#1}\mbox{}}
\fi
\ifx\subparagraph\undefined\else
\let\oldsubparagraph\subparagraph
\renewcommand{\subparagraph}[1]{\oldsubparagraph{#1}\mbox{}}
\fi

\date{}


\usepackage{enumitem}
\makeatletter
\newcommand{\xslalph}[1]{\expandafter\@xslalph\csname c@#1\endcsname}
\newcommand{\@xslalph}[1]{%
    \ifcase#1\or а\or б\or в\or г\or д\or e\or є\or ж\or з\or i%
    \or й\or к\or л\or м\or н\or о\or п\or р\or с\or т%
    \or у\or ф\or х\or ц\or ч\or ш\or ю\or я\or аа\or бб\or вв%
    \else\@ctrerr\fi%
}
\AddEnumerateCounter{\xslalph}{\@xslalph}{m}
\makeatother


\begin{document}


\newpage
\subsection{3.3. Динамічні масиви. Робота з вказівниками }
\setcounter{subsection}{1}


\begin{itemize}
\item
  Як можна створити лінійний динамічний масив та коректно завершити при
  цьому програму?
\item
  Що таке вказівники? Які операції визначені на вказівниках? Як
  проітеруватись по даному масиву за допомогою вказівника?
\item
  Як визначити динамічну матрицю за допомогою масиву вказівників та
  коректно її обробити?
\item
  Які функції та з якої бібліотеки використовуються на Сі для виділення
  памяті? В чому їх різниця? Що відбудеться якщо потрібної пам'яті не
  було ними виділено?
\item
  Які функції існують для очищення пам'яті? Що відбудеться, якщо їх не
  використовувати? Які ще проблеми виникають при некоректному очищенні
  чи його відсутності?
\end{itemize}

Задачі для аудіторної роботи

\begin{enumerate}
\def\labelenumi{\arabic{enumi})}
\item
  Ввести натуральне число $n$. Створити та ввести масив з $n$ дійсних чисел та
  підрахувати суму квадратів елементів цього масиву. 

\item
  Написати функцію, що вводить масив цілих чисел доки не введеться нуль
  через змінний аргумент та кількість елементів масиву повертається як
  результат роботи функції. Кількість елементів обмежена числом 100.
  Підрахувати кількість повних квадратів та кубів в цьому масиві.
\item
  Створити функцію, що вводить $n$-вимірний вектор($n$
  задається як аргумент функції), виділяючи відповідну
  пам'ять та функцію, що відповідно очищує пам'ять. Напишіть програму,
  що вводить два вектори, підраховує та створює як окремий масив їх
  різницю якщо це можливо, та в будь-якому варіанті коректно
  завершує програму без витоків пам'яті.
\item
  Створити функції, що коректно ініціалізують нулями та вводять з консолі 
дійсну квадратну $n$-вимірну матрицю ($n$ задається як аргумент функції), й
  функцію, що відповідно очищує пам'ять. Напишіть програму, що вводить
  дві матриці, підраховує та обчислює як окремий масив їх добуток, якщо
  це можливо, та в будь-якому варіанті коректно завершує програму без
  витоків пам'яті. Зробіть дану задачу:
  \begin{itemize}
  \item
представляючі матрицю у вигляді двовимірного масиву;
  \item
представляючі матрицю у вигляді лінійного масиву розміру $n^{2}$.
 \end{itemize}

\end{enumerate}

Задачі для самостійної роботи

\begin{enumerate}
\def\labelenumi{\arabic{enumi})}
\setcounter{enumi}{4}
\item
  Створити функцію, що вводить матрицю цілих чисел довільних
  розмірностей, виділяючи відповідну пам'ять (розміри масивів) та
  функцію, що відповідно очищує пам'ять. Напишіть функцію, що підраховує
  ранг матриці. Коректно протестуйте роботу цих функцій.
\item
  Створити функцію, що вводить матриці довільних розмірностей, виділяючи
  відповідну пам'ять та функцію, що відповідно очищує пам'ять. Напишіть
  програму, що вводить масив таких матриць, підраховує та створює як
  окремий масив добуток всього масиву матриць, якщо це можливо, та в
  будь-якому варіанті коректно завершує програму без витоків пам'яті.

\item
  Ввести натуральне число $n$. Створити та ввести масив з $n$ натуральних
 довгих чисел та підрахувати кількість ступенів двійки та трійки в цьому масиві.
\item
  Вирішіть завдання виконуючи наступні вимоги:

Сформувати динамічний двовимірний дійсний масив $N \times M$, заповнити його випадковими
числами або з консолі та вивести на екран. Виконати наступні дії, коректно оброблюючи 
всі можливі сценарії:

\begin{enumerate}[label=\xslalph*)]
\item
  додати рядок після заданого номеру $k$;
\item
  додати стовпець після заданого номеру $k$;
\item
  додати рядок в кінець матриці;
\item
  додати стовпець в кінець матриці;
\item
  додати рядок в початок матриці;
\item
  додати стовпець в початок матриці;
\item
  додати $k$ рядків в кінець матриці;
\item
  додати $k$ стовпців в кінець матриці;
\item
  додати $k$ рядків в початок матриці;
\item
  додати $k$ стовпців в початок матриці;
\item
  видалити рядок з номером $k$;
\item
  видалити стовпець з номером $k$;
\item
  видалити рядки, починаючи з рядка $k1$ і до рядка $k2$;
\item
  видалити стовпці, починаючи з стовпця $k$ і до стовпчика $k$;
\item
  видалити всі непарні рядки;
\item
  видалити всі парні стовпці;
\item
  видалити всі рядки, в яких є хоча б один нульовий елемент;
\item
  видалити всі стовпці, в яких всі елементи менші за 1;
\item
  видалити рядок, в якій знаходиться наймеший за модулем елемент матриці (
якщо їх декілька -- видалити усі);
\item
  додати рядок після кожного парного рядку матриці;
\item
  додати стовпець після кожного парного стовпця матриці;
\item
  додати $k$ рядків, починаючи з рядку за номером $m$;
\item
  додати $k$ стовпців, починаючи зі стовпчика за номером $m$;
\item
  додати рядок після рядка, що містить найбільший елемент;
\item
  додати стовпець після стовпця, що має найбільшу суму елементів;
\item
  додати рядок після рядка, що має найменше значення норми (суми квадратів)(
якщо їх декілька -- обираємо останній);
\item
  додати стовпець після стовпця, що містить найменший за модулем елемент (
якщо їх декілька -- обираємо перший);
\item
  видалити рядок і стовпець, на перетині яких знаходиться найбільший
  елемент матриці.
\end{enumerate}

\end{enumerate}

Додаткові задачі

\begin{enumerate}
\def\labelenumi{\arabic{enumi})}
\setcounter{enumi}{8}
\item
  Користувачу надається можливість декілька разів вводити розмірність
  вектору дійсних чисел та самі ці значення. Після кожного вводу
  потрібно підрахувати середнє арифметичне та дисперсію всіх введених
  значень.
\item
  Петя та Вася кожен день на протязі $N$ днів вимірюють
  декілька (від 0 до 1000) разів температуру повітря (хоча інколи хтось
  може забути це зробити). Створіть програму, що дозволить їм ввести ці
  результати за кожен день спостережень та підрахує середню температуру
  кожного з цих днів, де сумарна кількість вимірювань була більше 1.
  Програма повинна передбачити, що після вводу цих $N$ днів вони можуть
  захотіти ввести наступні $M$ днів таких спостережень. Передбачте
  можливість коректного завершення при нестачі ресурсів комп'ютера для
  зберігання та обробки даних.
\item
 В масиві натуральних чисел A{[}N{]} всі числа є меншими 16. Напишить
  функцію, що зберігає дані цього масиву у масиві $N/2$ чисел типу
  uint8\_t (тобто в кожному числі uint8\_t зберігається два числа масиву
  A{[}i{]}).
\item
  В масиві натуральних чисел A{[}N{]} всі числа є меншими 64. Напишить
  функцію, що зберігає дані цього масиву у масиві $[N*4/3]$ чисел типу
  uint8\_t (тобто в кожних трьох числах uint8\_t зберігається чотири
  числа масиву A{[}i{]}).
\item
  В масиві натуральних чисел A{[}N{]} всі числа є меншими \(2^{k}\).
  Знайдіть це число $k$ та напишить функцію, що зберігає цей масив в $N*k$
  біт найбільш економічним чином( int A{[}3{]}, k=5 → uint8 B{[}2{]}
  ,тобто використовує 16 біт, або int A{[}8{]}, k=14 → uint16 B{[}7{]} ,
  тобто використовує 112 біт) та функцію що обратно повертає числа з
  масиву B у масив A.
\end{enumerate}

\end{document}
