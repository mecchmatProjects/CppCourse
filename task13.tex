\documentclass[]{article}
\usepackage{lmodern}
\usepackage{amssymb,amsmath}
\usepackage{ifxetex,ifluatex}


\usepackage[utf8]{inputenc}
\usepackage[english,russian,ukrainian]{babel}

\usepackage{fixltx2e} % provides \textsubscript
\ifnum 0\ifxetex 1\fi\ifluatex 1\fi=0 % if pdftex
  \usepackage[T1]{fontenc}
  \usepackage[utf8]{inputenc}
\else % if luatex or xelatex
  \ifxetex
    \usepackage{mathspec}
  \else
    \usepackage{fontspec}
  \fi
  \defaultfontfeatures{Ligatures=TeX,Scale=MatchLowercase}
\fi
% use upquote if available, for straight quotes in verbatim environments
\IfFileExists{upquote.sty}{\usepackage{upquote}}{}
% use microtype if available
\IfFileExists{microtype.sty}{%
\usepackage{microtype}
\UseMicrotypeSet[protrusion]{basicmath} % disable protrusion for tt fonts
}{}
\usepackage[unicode=true]{hyperref}
\hypersetup{
            pdfborder={0 0 0},
            breaklinks=true}
\urlstyle{same}  % don't use monospace font for urls
\usepackage{graphicx,grffile}
\makeatletter
\def\maxwidth{\ifdim\Gin@nat@width>\linewidth\linewidth\else\Gin@nat@width\fi}
\def\maxheight{\ifdim\Gin@nat@height>\textheight\textheight\else\Gin@nat@height\fi}
\makeatother
% Scale images if necessary, so that they will not overflow the page
% margins by default, and it is still possible to overwrite the defaults
% using explicit options in \includegraphics[width, height, ...]{}
\setkeys{Gin}{width=\maxwidth,height=\maxheight,keepaspectratio}
\IfFileExists{parskip.sty}{%
\usepackage{parskip}
}{% else
\setlength{\parindent}{0pt}
\setlength{\parskip}{6pt plus 2pt minus 1pt}
}
\setlength{\emergencystretch}{3em}  % prevent overfull lines
\providecommand{\tightlist}{%
  \setlength{\itemsep}{0pt}\setlength{\parskip}{0pt}}
\setcounter{secnumdepth}{0}
% Redefines (sub)paragraphs to behave more like sections
\ifx\paragraph\undefined\else
\let\oldparagraph\paragraph
\renewcommand{\paragraph}[1]{\oldparagraph{#1}\mbox{}}
\fi
\ifx\subparagraph\undefined\else
\let\oldsubparagraph\subparagraph
\renewcommand{\subparagraph}[1]{\oldsubparagraph{#1}\mbox{}}
\fi

\date{}


\usepackage{enumitem}
\makeatletter
\newcommand{\xslalph}[1]{\expandafter\@xslalph\csname c@#1\endcsname}
\newcommand{\@xslalph}[1]{%
    \ifcase#1\or а\or б\or в\or г\or д\or e\or є\or ж\or з\or i%
    \or й\or к\or л\or м\or н\or о\or п\or р\or с\or т%
    \or у\or ф\or х\or ц\or ч\or ш\or ю\or я\or аа\or бб\or вв%
    \else\@ctrerr\fi%
}
\AddEnumerateCounter{\xslalph}{\@xslalph}{m}
\makeatother


\begin{document}


\newpage
\subsection{12. Введення/виведення на С++. Робота з текстовими файлами на Сі++}
\setcounter{subsection}{1}


\begin{itemize}

\item
  Як використовувати бібліотеки Сі на Сі++? Що потрібно для того щоб код
  на Сі працював так само на Сі++?
\item
  Яка різниця булевого типу та його використання на Сі та Сі++?
\item
  Як вивести в Сі++ використовуючи потоки виведення дійсне число з
  заданою точністю? В науковому представленні? З заданою шириною?
\item
  Як записати у текстовий файл масив цілих чисел через кому у якості
  роздільника та прочитати потім цей масив?
\item
  Що таке перевантаження функцій та навіщо воно може бути потрібно?
\item
  Що таке new та new{[}{]}? Коли потрібно перше та коли друге?
\item
  В чому різниця між new та malloc?
\item
  Як очищувати пам'ять після new та new{[}{]}?
\end{itemize}

Задачі для аудиторної роботи

\begin{enumerate}
\def\labelenumi{\arabic{enumi})}
\item
  Ввести в двох різних рядках послідовно два дійсних числа x та y та
  обчислити значення x в ступені y. Результат вивести в десятковому та
  науковому представленні.
\item
  На терміналі вводяться $10*n$ цифр. Перші 10 цифр -- це перше натуральне
  число, наступні 10 -- друге і так далі. Введіть всі ці числа в масив
  розміру $n$ та обчисліть і виведіть їх суму (вважайте що сума влазить в
  точність unsigned long long).
\item
  Вивести на екран таблицю для всіх чисел від 1 до $n$ 
 (організувати прицьому переноси на нові рядки для заданої довжини), 
  слідкуючи, щоб виведення було рівним та кількість цифр після коми була або 0 або 2:\\
+++++++++++++++ +++++++++++\\
+число ++ \ 1 + \ 2 \ + \ 3 \ + \ 4 \ + 5\\
++++++++++++++++++++++++++++\\
+корінь+ 1 +1.44 + 1.69 + 2\\
++++++++++++++++ ++++++++++

\item
  Ввести з текстового файлу та з консолі натуральне число $n$ та масиви з
  $n$ цілих чисел \(\left\{ m_{i} \right\}_{i = 1}^{n}\) та дійсних чисел
  \(\left\{ x_{i} \right\}_{i = 1}^{n}\). Обчислить та виведіть у файл
  числа \(\left\{ x_{i}^{m_{i}} \right\}_{i = 1}^{n}\).
\item
  Вхідний потік заданий текстовим файлом містить набір цілих чисел $A_i (0
  \le A_i \le 10^{18}$), відділений один від іншого довільною кількістю пробілів
  і переводів рядків. Розмір вхідного потоку не перевищує 256 КБ. Для
  кожного числа $A_i$, починаючи з останнього та завершуючи першим, в
  окремому рядку вивести його квадратний корінь не менш ніж з чотирма
  знаками після десяткової крапки.

Приклад:

\textbf{Вхід:}

1427 \ \ 0

 \ \ 876652098643267843

\ 5276538

\textbf{Вихід: }

2297.0716

936297014.1164

0.0000

37.7757

\end{enumerate}

Задачі для самостійної роботи

\begin{enumerate}
\def\labelenumi{\arabic{enumi})}
\setcounter{enumi}{5}
\item
  Ввести декілька (невідомо зазделегідь скільки) дійсних числа записаних
  через коми та обчислити значення функції $log()$ для кожного з них. Якщо
  значення виходить за межі області вивести слово ``None'', для інших
  значень результат вивести в науковому та десятковому представленні
  шириною 5 символів.
\item
  Три додатніх дійсні числа вводяться як рядок вигляду \\
  А=ххх.ххх, B=xxExxx C=xxx.xxxx\\
  Обчисліть їх середнє гармонічне та виведіть у науковому та звичайному
  форматі.
\item
  Ввести дійсне число від -10000 до 10000 та вивести його $k$-ту ступінь
  ($|k|<10$) з точністю до 20 знаків до десяткової коми та 4
  знаками після десяткової коми (нуль залишається нулем завжди).
\end{enumerate}

\textbf{Організуйте роботу з текстовим файлом. Вихідні файли не передбачають
  зміни. Змінені дані збережіть в іншому файлі.}

\begin{enumerate}
\def\labelenumi{\arabic{enumi})}
\setcounter{enumi}{8}
\item
  Дано два текстові файли з іменами Name1 і Name2. Додати в кінець
  кожного рядка файлу Name1 відповідний рядок файлу Name2. Якщо файл
  Name2 коротший файлу Name1, то виконайте перехід до початку файлу
  Name2.
\item
  Організувати текстовий файл, що складається з N рядків. Визначити
  максимальний і мінімальний розмір рядків в файлі і вивести їх в інший
  файл.
\item
  Дан текстовий файл з ім'ям NameT. Підрахувати число повторень в ньому
  малих латинських літер ('a' - 'z') і створити файл з ім'ям NameS,
  рядки якого мають вигляд: "\textless{}літера\textgreater{} -
  \textless{}число повторень даної літери\textgreater{}". Літери,
  відсутні в тексті, в файл не включати. Рядки впорядкувати за спаданням
  кількості повторень літер, а при однаковій кількості повторень - по
  зростанню кодів літер.
\item
  Дан символ с (прописна латинська літера) і текстовий файл. Створити
  текстовий файл, який містить всі слова з вихідного файлу, що
  починаються цією літерою (як великої, так і малої). Розділові знаки,
  розташовані на початках і в кінцях слів, не враховувати. Якщо вихідний
  файл не містить відповідних слів, залишити результуючий файл порожнім.
\end{enumerate}

\textbf{Організуйте роботу з текстовим файлом. Вхідний файл потрібно
змінити згідно вказаних умов, тобто вхідний та вихідні файли
співпадають.}

\begin{enumerate}
\def\labelenumi{\arabic{enumi})}
\setcounter{enumi}{12}
\item
  Дано число N і текстовий файл. Видалити з файлу рядки з номерами,
  кратними N. Порожні рядки не враховувати і не видаляти. Якщо рядки з
  необхідними номерами відсутня, то залишити файл без змін. Зміна
  вивести в другий файл.
\item
  Дан текстовий файл, що містить текст, вирівняний по лівому краю
  (довжина кожного рядка не перевищує 50 символів). Вирівняти його по
  центру, додавши в початок кожної непорожній рядки необхідну кількість
  прогалин. Рядки непарної довжини перед центруванням доповнювати зліва
  прогалиною. Вирівняний текст записати в інший файл.
\item
  Організувати текстовий файл, що складається з N рядків. Перетворити
  файл, видаливши в кожній його рядку зайві пробіли. Зміни вивести в
  другий файл.
\item
  Дан файл з текстом із символів латинського алфавіту. Зашифрувати файл,
  виконавши циклічний зсув кожної букви вперед на n позицій в алфавіті.
  Розділові знаки і пропуски не змінювати.
\item
  Дано числа N1, N2 і текстовий файл. Видалити з файлу рядки з номерами
  між N1, N2, не включаючи меж. Зміни вивести в другий файл. Якщо
  виконати видалення неможливо, видайте про це повідомлення на екран і в
  вихідний файл.
\item
  Дан файл з текстом із символів латинського алфавіту, цифр та знаків.
  Замініть всі цифри їх назвами на англійській мові.
\item
  Створити текстовий файл f, що складається з N рядків. Після цього
  створити файли h і g. У файл h записати рядки файлу f непарної
  довжини, в файл g парної довжини.

\item
 Визначити функцію, яка:
\begin{itemize}
\item підраховує кількість порожніх рядків;
\item обчислює максимальну довжину рядків текстового файлу.
\end{itemize}

\item Визначити процедуру виведення:
\begin{itemize}
\item усіх рядків текстового файлу;
\item рядків, які містять більше 60 символів.
\end{itemize}

\item
Визначити функцію, що визначає кількість рядків текстового файлу,
які:
\begin{itemize}
\item починаються із заданого символу;
\item закінчуються заданим символом;
\item починаються й закінчуються одним і тим самим символом;
\item що складаються з однакових символів.
\end{itemize}

\item
В даному текстовому файлі знаходиться англомовний текст. Вирівняйте
його по лівий та правий границі так щоб розподіл слів у рядках був
найбільш рівномірним.

\item
Визначити процедуру, яка переписує до текстового файлу G усі 
рядки текстового файлу F:
\begin{itemize}
\item із заміною в них символу '0' на '1', і навпаки;
\item в інвертованому вигляді.
\end{itemize}

\item
Визначити процедуру пошуку найдовшого рядка в текстовому файлі.
Якщо таких рядків кілька, знайти перший із них.
\item
Визначити процедуру, яка переписує компоненти текстового 
файлу F до файлу G, вставляючи до початку кожного рядка один символ пропуску.
Порядок компонент не має змінюватися.
\item
У текстовому файлі записано непорожню послідовність дійсних чисел,
які розділяються пропусками. Визначити функцію обчислення найбільшого з
цих чисел.

\item
У текстовому файлі F записано послідовність цілих чисел, які розділяються пропусками. 
Визначити процедуру запису до текстового файлу g усіх додатних чисел із F.

\item
У текстовому файлі кожний рядок містить кілька натуральних чисел, які розділяються пропусками.
Числа визначають вигляд геометричної фігури (номер) та її розміри. Прийнято такі домовленості:
\begin{itemize}
\item
відрізок прямої задається координатами своїх кінців і має номер 1;
\item
прямокутник задається координатами верхнього лівого й нижнього правого кутів і має номер 2;
\item
коло задається координатами центра й радіусом і має номер 3.
\end{itemize}

Визначити процедури обчислення:
\begin{itemize}
\item відрізка з найбільшою довжиною;
\item прямокутника з найбільшим периметром;
\item кола з найменшою площею.
\end{itemize}


\item 
У файлі записані координати точок на площині задані парою цілих
чисел. Точки записуються в форматі : ( х1 , х2 ) (х1 , х2) , \ldots{} -
саме так через коми та дужки. Створити файл, в якому будуть записані
координати всіх відрізків з точок цього файлу, при цьому ці відрізки
відсортовані за зростанням довжини.

\item 
У файлі записані координати Точок в просторі задані трійкою цілих
чисел. Точки записуються в форматі : х1 , х2 , х3 ; х1 , х2, х3;
\ldots{} Знайти відрізок з точок цього файлу, що має найбільшу довжину.

\item 
У файлі записані координати матеріальних точок на площині задані
парою цілих чисел та масою(дійсне число). Точки записуються в форматі :
{[}х1 , y1, m1 {]}, {[}х2 , y2, m2{]} , \ldots{} \textbf{- саме так
через коми та дужки. Знайдіть дві точки} з найбільшим важілем сили (m*(х
+y)).

\item 
У файлі записані дати , що задані трійкою цілих чисел у форматі
(чч1./мм1/рр1),(чч2./мм2/рр2), \ldots{} - саме в такому форматі.
Створити файл, в якому будуть записано найстарша та найсвіжіша дати
(врахуйте, що роки дат з 1951 по 2049).

\end{enumerate}

Додаткові задачі:
\begin{enumerate}
\def\labelenumi{\arabic{enumi})}
\setcounter{enumi}{33}

\item  Розглянемо послідовність чисел \(a_{i}\) , i = 0, 1, 2, \ldots{}, що
задовольняють умовам:

\(a_{0} = 0\), \(a_{1} = 1\); та \(a_{2i} = a_{i}\) і
\(a_{2i + 1} = {2a}_{i} + 1\) для кожного $i = 1, 2, 3, \ldots{} $.

Напишіть програму, яка для заданого значення $n$ знаходить максимальне
серед чисел \(a_{0},a_{1},\cdots,a_{n}\). Вхідні дані складаються з
декількох тестів (не більше 10). Кожен тест - рядок, в якому записано
ціле число n ($1 \le n \le 99 999$). В останньому рядку вхідних даних записано
число 0. Для кожного $n$ у виводі запишіть максимальне значення.

\item
Створити текстовий (.txt) файл з 100,000,000 рядків з числами
в діапазоні від 0 до 99,999,999. Формат чисел - 8 нулів (1 = 00000001, 65535 = 00065535) , діапазон
від 0 до 99999999, всі числа розташовані в випадковому порядку без
повторів (кожен рядок -- унікальне число).

\emph{Приклад.}

\emph{00306453 }

\emph{99645283 }

\emph{70000021 }

\emph{06847127 }
\end{enumerate}

\end{document}
