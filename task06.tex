\documentclass[]{article}
\usepackage{lmodern}
\usepackage{amssymb,amsmath}
\usepackage{ifxetex,ifluatex}


\usepackage[utf8]{inputenc}
\usepackage[english,russian,ukrainian]{babel}

\usepackage{fixltx2e} % provides \textsubscript
\ifnum 0\ifxetex 1\fi\ifluatex 1\fi=0 % if pdftex
  \usepackage[T1]{fontenc}
  \usepackage[utf8]{inputenc}
\else % if luatex or xelatex
  \ifxetex
    \usepackage{mathspec}
  \else
    \usepackage{fontspec}
  \fi
  \defaultfontfeatures{Ligatures=TeX,Scale=MatchLowercase}
\fi
% use upquote if available, for straight quotes in verbatim environments
\IfFileExists{upquote.sty}{\usepackage{upquote}}{}
% use microtype if available
\IfFileExists{microtype.sty}{%
\usepackage{microtype}
\UseMicrotypeSet[protrusion]{basicmath} % disable protrusion for tt fonts
}{}
\usepackage[unicode=true]{hyperref}
\hypersetup{
            pdfborder={0 0 0},
            breaklinks=true}
\urlstyle{same}  % don't use monospace font for urls
\usepackage{graphicx,grffile}
\makeatletter
\def\maxwidth{\ifdim\Gin@nat@width>\linewidth\linewidth\else\Gin@nat@width\fi}
\def\maxheight{\ifdim\Gin@nat@height>\textheight\textheight\else\Gin@nat@height\fi}
\makeatother
% Scale images if necessary, so that they will not overflow the page
% margins by default, and it is still possible to overwrite the defaults
% using explicit options in \includegraphics[width, height, ...]{}
\setkeys{Gin}{width=\maxwidth,height=\maxheight,keepaspectratio}
\IfFileExists{parskip.sty}{%
\usepackage{parskip}
}{% else
\setlength{\parindent}{0pt}
\setlength{\parskip}{6pt plus 2pt minus 1pt}
}
\setlength{\emergencystretch}{3em}  % prevent overfull lines
\providecommand{\tightlist}{%
  \setlength{\itemsep}{0pt}\setlength{\parskip}{0pt}}
\setcounter{secnumdepth}{0}
% Redefines (sub)paragraphs to behave more like sections
\ifx\paragraph\undefined\else
\let\oldparagraph\paragraph
\renewcommand{\paragraph}[1]{\oldparagraph{#1}\mbox{}}
\fi
\ifx\subparagraph\undefined\else
\let\oldsubparagraph\subparagraph
\renewcommand{\subparagraph}[1]{\oldsubparagraph{#1}\mbox{}}
\fi

\date{}


\usepackage{enumitem}
\makeatletter
\newcommand{\xslalph}[1]{\expandafter\@xslalph\csname c@#1\endcsname}
\newcommand{\@xslalph}[1]{%
    \ifcase#1\or а\or б\or в\or г\or д\or e\or є\or ж\or з\or i%
    \or й\or к\or л\or м\or н\or о\or п\or р\or с\or т%
    \or у\or ф\or х\or ц\or ч\or ш\or ю\or я\or аа\or бб\or вв%
    \else\@ctrerr\fi%
}
\AddEnumerateCounter{\xslalph}{\@xslalph}{m}
\makeatother


\begin{document}


\newpage
\subsection{ 6. Бітові операції }
\setcounter{subsection}{1}


\begin{itemize}
\item
  Що таке та які бітові операції існують? Який пріоритет цих операцій?
\item
  Чому дорівнюють наступні вирази:
  3 \textless{} \textless{} 2, 5 \textgreater{} \textgreater{} 2, 5 \& 3, n \& 1, n | 1, n\textasciicircum n, \textasciitilde{}0.
\item
  Як знайти значення самого лівого біту? Самого правого? Третього зліва?
  Як встановити 5-й байт зліва в 1? В нуль?
\item
  Для яких типів Сі краще не застосовувати бітові операції?
\item
  Який нюанс для першого біту є при використанні бітових операцій для
  цілого типу С/С++?
\end{itemize}

Аудиторні задачі

\begin{enumerate}
\def\labelenumi{\arabic{enumi})}
\item
  Ввести натуральне 8-бітове число $n$ і вивести $2^{n}$, використовуючи бітові операції.
\item
  Ввести ціле число $n$ та натуральне $k$ і вивести ціле число, яке у якого
  $k$-й біт встановлений в 1, а всі інші біти збігаються з бітами числа $n$
  на тих же позиціях. Наприклад, якщо введені 9 і 1, відповіддю буде 11.
\item
  Вести натуральне довге число $M$. Встановіть її біт
  з номером $j$ рівним нулеві та виведіть отримане число в десятковому та
  шістнадцятковому вигляді.
\item
  Поміняйте місцями перші 8 біт та останні 8 біт натурального числа,
  виведіть отримане число в десятковому та шістнадцятковому вигляді.
\item
  Підрахуйте найбільшу кількість одиничок серед бітів даного числа, що
  йдуть підряд.
\item
  Описати словами результат наступного виразу: x \& (x-1).
\item
  Описати словами результат наступного виразу: x \& (-x).
\item
  Напишіть функцію що визначає до якої архітектури (big, high, little
  endian) належить даний комп'ютер.
\end{enumerate}

Самостійна

\begin{enumerate}
\def\labelenumi{\arabic{enumi})}
\item
  Ввести натуральне(32-бітне) число $M$. Встановіть її $j$-тий рівним нулеві
  та виведіть отримане число виведіть отримане число в десятковому та
  шістнадцятковому вигляді.
\item
  Визначить номер першого значущого зліва та зправа біта натурального
  числа $M$.
\item
  Поміняйте місцями перші 8 біт та останні 8 біт натурального числа
  (розмір в бітах якого вважаємо невідомим до вводу) та виведіть
  отримане число в десятковому та шістнадцятковому вигляді.
\item
  Ввести натуральне 64-бітне число $M$. Встановіть її ліві $n$ біт рівним
  нулеві та виведіть отримане число. Встановіть її праві $n$ біт рівним
  нулеві та виведіть отримане число в десятковому та вісімковому
  вигляді. Розвяжить задачу для типу $M$ unsigned та long long unsigned.
\item
  Ввести натуральне число $M$. Поміняйте місцями біти її двійкового запису
  з номерами $i$ та $j$ (що теж вводяться) та виведіть отримане число в
  десятковому та шістнадцятковому вигляді.
\item
  Знайдіть кількість значущих (не рівних 0) бітів натурального
  32-бітного числа.
\item
  За допомогою лище бітових операцій та операції декременту зясуйте чи є
  дане натуральне число ступінню двійки. Спробуйте з циклом та без
  циклу. (Підказка: подумайте, як виглядає бітове представлення
  декременту ступеню двійки, та використайте далі конюнкцію).
\item
  Ввести натуральні 32-бітні числа $M$ та $N$ та визначить скільки в них
  спільних одиничок бітового представлення. Визначить скільки в цих
  числах взагалі співпадає бітів.
\item
  Виведіть бітове (двійкове) представлення натурального числа.
\item
  Інвертуйте (тобто прочитайте зліва направо) бітове представлення
  даного числа та виведіть двійкове представлення та десяткове для цієї
  інверсії.
\item
  Ввести ціле число n (однобайтове) і вивести число, отримане в
  результаті циклічного зсуву числа n на один розряд вліво, тобто
  старший біт зсунитий в позицію молодшого, а всі інші біти зсуваються
  на один розряд вліво. Наприклад, якщо введено 130, відповіддю буде 5.
\item
  Визначити, скільки разів зустрічається 11 в двійковому поданні цілого
  додатнього числа (в двійковому поданні 11110111 воно зустрічається 5
  разів).
\item
  Викреслити $i$-й біт з двійкового представлення натурального числа
  (молодші $i$ бітів залишаються на місці, старші зсуваються на один
  розряд вправо). Наприклад, якщо введені 11 і 2, відповіддю буде 7.
\item
  Написати функцію, результатом якого є дане значення $x$, у якого молодший нульовий біт та найстарший
  біт встановлені в 1.
\item
  Написати функцію, результатом якого є дане значення $x$, у якого все
  біти встановлені в 1, крім молодших $n$ бітів.
\item
  Підрахуйте кількість нулів серед бітів даного числа.
\item
  Знайдіть номер найстаршого значущого біта в даному 32-бітному числі.
\item
  Напишіть функцію, що визначає чи два натуральних числа не мають
  одиничних бітів на однакових позиціях.
\item
  Напишіть функцію, що визначає чи два натуральних числа не мають
  нульових бітів на однакових позиціях.
\item
  Напишіть функцію, що визначає чи два натуральних числа не мають
  однакових бітів на однакових позиціях.
\end{enumerate}


\end{document}
