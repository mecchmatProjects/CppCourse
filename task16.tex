\documentclass[]{article}
\usepackage{lmodern}
\usepackage{amssymb,amsmath}
\usepackage{ifxetex,ifluatex}


\usepackage[utf8]{inputenc}
\usepackage[english,russian,ukrainian]{babel}

\usepackage{fixltx2e} % provides \textsubscript
\ifnum 0\ifxetex 1\fi\ifluatex 1\fi=0 % if pdftex
  \usepackage[T1]{fontenc}
  \usepackage[utf8]{inputenc}
\else % if luatex or xelatex
  \ifxetex
    \usepackage{mathspec}
  \else
    \usepackage{fontspec}
  \fi
  \defaultfontfeatures{Ligatures=TeX,Scale=MatchLowercase}
\fi
% use upquote if available, for straight quotes in verbatim environments
\IfFileExists{upquote.sty}{\usepackage{upquote}}{}
% use microtype if available
\IfFileExists{microtype.sty}{%
\usepackage{microtype}
\UseMicrotypeSet[protrusion]{basicmath} % disable protrusion for tt fonts
}{}
\usepackage[unicode=true]{hyperref}
\hypersetup{
            pdfborder={0 0 0},
            breaklinks=true}
\urlstyle{same}  % don't use monospace font for urls
\usepackage{graphicx,grffile}
\makeatletter
\def\maxwidth{\ifdim\Gin@nat@width>\linewidth\linewidth\else\Gin@nat@width\fi}
\def\maxheight{\ifdim\Gin@nat@height>\textheight\textheight\else\Gin@nat@height\fi}
\makeatother
% Scale images if necessary, so that they will not overflow the page
% margins by default, and it is still possible to overwrite the defaults
% using explicit options in \includegraphics[width, height, ...]{}
\setkeys{Gin}{width=\maxwidth,height=\maxheight,keepaspectratio}
\IfFileExists{parskip.sty}{%
\usepackage{parskip}
}{% else
\setlength{\parindent}{0pt}
\setlength{\parskip}{6pt plus 2pt minus 1pt}
}
\setlength{\emergencystretch}{3em}  % prevent overfull lines
\providecommand{\tightlist}{%
  \setlength{\itemsep}{0pt}\setlength{\parskip}{0pt}}
\setcounter{secnumdepth}{0}
% Redefines (sub)paragraphs to behave more like sections
\ifx\paragraph\undefined\else
\let\oldparagraph\paragraph
\renewcommand{\paragraph}[1]{\oldparagraph{#1}\mbox{}}
\fi
\ifx\subparagraph\undefined\else
\let\oldsubparagraph\subparagraph
\renewcommand{\subparagraph}[1]{\oldsubparagraph{#1}\mbox{}}
\fi

\date{}


\usepackage{enumitem}
\makeatletter
\newcommand{\xslalph}[1]{\expandafter\@xslalph\csname c@#1\endcsname}
\newcommand{\@xslalph}[1]{%
    \ifcase#1\or а\or б\or в\or г\or д\or e\or є\or ж\or з\or i%
    \or й\or к\or л\or м\or н\or о\or п\or р\or с\or т%
    \or у\or ф\or х\or ц\or ч\or ш\or ю\or я\or аа\or бб\or вв%
    \else\@ctrerr\fi%
}
\AddEnumerateCounter{\xslalph}{\@xslalph}{m}
\makeatother


\begin{document}


\newpage
\subsection{15. Робота з класами. Наслідування та поліморфізм.}
\setcounter{subsection}{1}


\begin{itemize}
\item
  Що таке перевантаження методів? Чому воно зручно в мовах зі строгою
  типізацією?
\item
  Чим перевантаження операторів відрізняється від перевантаження інших
  методів?
\item
  Які оператори не можна перевантажувати? Коли перевантаження операторів
  може бути небезпечним?
\item
  Чому при перевантаженні операторів вводу-виводу нам потрібно ключове
  слово friend?
\item
  Які типи наслідування є на Сі++ та яка між ними різниця?
\item
  Поясніть на прикладі, що таке раннє та пізнє зв'язування
\item
  Що таке чисто віртуальний клас та чисто віртуальний метод? Коли вони
  потрібні?
\item
  Що таке віртуальний деструктор, та чому він потрібний?
\item
  Як реалізувати множинне наслідування на Сі++?
\item
  Що робити та які шляхи правильного множинного наслідування якщо й
  класи батьки й клас-син мають метод з однаковою назвою? Що зміниться,
  якщо це не метод, а перевантажений оператор?
\end{itemize}

Задачі для аудиторної роботи

\begin{enumerate}
\def\labelenumi{\arabic{enumi})}

\item
Клас Person описано таким чином:

\textbf{class} \textbf{Person\{} //Клас Особа

string name; //прізвище

unsigned byear\textbf{;//}рік народження

public:

\textbf{int} input()\textbf{\{} //ввести особу

\textbf{cin\textgreater{}\textgreater{}}name;

\textbf{cin\textgreater{}\textgreater{}byer;}

\textbf{\}}

\textbf{void} \textbf{print()\{ //}вивести особу

\textbf{cout\textless{}\textless{}}name\textless{}\textless{}'',''\textless{}\textless{}byear\textbf{\textless{}\textless{}endl;}

\}

  Описати клас Знайомий на базі класу Person. У цьому класі повинно бути 
як мінімум одне додаткове поле «номер
телефону» а також методи введення та виведення інформації про знайомого. 
Використати цей клас для побудови класу телефонного довідника (кількість
знайомих обмежена числом 100). Передбачити дії: створення довідника, додавання запису про знайомого,
пошуку номера телефону за прізвищем та заміни номера телефону. 
Телефонний довідник зберігає дані про знайомих у файлі.
\emph{\emph{Вказівка}}: телефонний довідник представити у вигляді класу
що зчитує дані з (текстового) файлу.

\item
  На базі класу Точка на площині створіть клас Точка3Д (точка
  в просторі. Реалізуйте методи введення, виведення. Аналогічно на базі
  Відрізка2Д реалізуйте клас Відрізок3Д. Реалызуйте методи
  введення та виведення, визначення довжини відрізка та
  визначення чи перетинаються 2 відрізка.

\end{enumerate}

Задачі для самостійної роботи

\begin{enumerate}
\def\labelenumi{\arabic{enumi})}
\setcounter{enumi}{2}
\item
  Описати клас Пасажир на базі класу Person. Клас містить дані про місце
  відправлення та місце слідування, а також місце пасажира. Створіть
  клас Каса, який дозволяє додавати та виводити інформацію про
  Пасижирів, містить методи пошуку по прізвищу, місцям відправлення,
  прибуття та місцю. Також серед заданого масиву місць у потягу знайдіть
  місце яке не зайняте (у випадку якщо таких місць декілька -- виведіть
  найменше за значенням, якщо їх немає відповідне повідомлення).

  \emph{\emph{Вказівка}}: інформацію про пасажирів представити у вигляді
бінарного файлу.

\item
  Описати клас Студент на базі класу Person.

У класі Студент повинна бути інформація про оцінки отримані ним протягом
сесії (за 5-ти бальною та 100 бальною шкалами).

Скласти програму для обчислення нарахованої студентам стипендії в
залежності від результатів сесії:

\begin{itemize}
\item
  За старим підходом нарахування стипендії (середній бал за всі іспити
  має бути не меншим ніж 4 за 5-ти бальною шкалою).
\item
  З новим підходом нарахування стипендії (стипендію отримують 40\% від
  загального числа студентів, які є найкращими по рейтингу)
\end{itemize}

\emph{\emph{Вказівка}}: інформацію про студентів представити у вигляді
масиву. Дані зчитувати з клавіатури.

\item
  Реалізувати клас СЛОВО, який має члени типу Рядок: ПРИСТАВКА,
  ПРИСТАВКА2, КОРІНЬ, СУФІКС, ЗАКІНЧЕННЯ (клас повинен мати геттери та
  сеттери).

Створіть наслідники цього класу: ГЛАГОЛ, ІМЕННИК, ПРИКМЕТНИК.

Реалізуйте для них методи: Род, Число, Лице, Відмінок -- які будуть
відповідним чино змінювати (якщо це можливо) дане слово.

Створіть декілька слів, що є екземплярами ГЛАГОЛу, ІМЕННИКу, ПРИКМЕТНИКу
та виконайте відповідні методи для них щоб можна було побачити
результат.

\item
  Реалізувати наступні класи:

Створити клас \textbf{Фігура}, який є базовим.
\begin{itemize}
\item
Описати клас \textbf{Прямокутник}. Сторони прямокутника паралельні осям
координат. Для прямокутника задані лівий верхній кут та довжини сторін.
Описати методи отримання довжини кожної з сторін, площі прямокутника,
периметру, чи перетинаються 2 прямокутника, координати центру мас. 
\item
Описати клас \textbf{Трикутник}. Основа трикутника паралельна осі
\emph{x} координат. Для трикутника задані ліва нижня координата,
довжина основи та 2 кути спільні з основою. Описати методи отримання довжини кожної зі сторін.
Описати методи отримання площі, периметру, координати центру мас.  
\item
Описати клас \textbf{Еліпс}. Для нього є заданими координати фокусів та радіуси.
Описати методи отримання геометричних характеристик. Описати методи
отримання довжини радіусів, площі, периметру, координати центру мас. 
\end{itemize}

Скласти програму створення заданої кількості фігур та знаходження їх спільного центру мас.


\item
Створити клас \textbf{Фігура}, який є базовим.  Опишіть класи для 
таких геометричних фігур та реалізуйте зазначені методи:
\begin{itemize}
\item
  Клас Трапеція. Основи трапеції паралельні вісі Ох. У цьому класі реалізуйте операції 
знаходження периметра і площі, методи переміщення та повороту.
\item
  Клас Паралелограм. Основи паралелограму паралельні вісі Ох. 
У цьому класі реалізуйте операції знаходження периметра і площі, 
методи переміщення та повороту.
\item
  Клас Круг. Реалізуйте методи відшукання площі круга, довжини кола,
  методи переміщення та повороту.
\end{itemize}
Скласти програму створення заданої кількості фігур, їх переміщення так щоб в них не було
перетінів та знаходження їх сумарної площі та периметру. 
Знайдіть фігуру з найбільшою площею.


\item

Створити клас \textbf{Фігура}, який є базовим.  Опишіть класи для 
таких геометричних фігур та реалізуйте зазначені методи:
\begin{itemize}
\item
Клас \textbf{Прямокутник}. 
Для прямокутника задані лівий верхній кут та правий нижній кут.
Описати методи отримання довжини кожної з сторін, площі прямокутника,
периметру. 
\item
Клас \textbf{ Трикутник}, що містить масив з 3 вершин. 
Описати методи отримання довжини кожної з сторін, площі прямокутника,
периметру. 
\item
Клас \textbf{ П'ятикутник}, що містить масив вершин. 
Реалізуйте метод перевірки чи є цей п'ятикутник опуклим.
\item
Клас \textbf{ Багатокутник}. 
Реалізуйте метод перевірки чи є цей багатокутник опуклим.
\end{itemize}
Дано список фігур вищенаведених класів. Знайдіть всі опуклі багатокутники.
Серед фігуру, що має найменший периметр.

\item

Створити клас \textbf{Фігура3D}, який є базовим.  Опишіть класи для 
таких геометричних фігур та реалізуйте зазначені методи:
\begin{itemize}
\item
  Клас Паралелипипед. Реалізуйте методи пошуку площі бічної поверхні і
  об'єму.
\item
  Клас Піраміда(трикутна). Реалізуйте методи пошуку площі бічної поверхні і
  об'єму.
\item
  Клас Піраміда(прямокутна). Реалізуйте методи пошуку площі бічної поверхні і
  об'єму.
\end{itemize}
Введіть масив фігур та підрахйте їх сумарний об'єм та сумарну площу всіх граней 
та загальну кількість вершин.

\item
Створити клас Лінійне рівняння для лініного рівняння з методом пошуку дійсного розвязку.
Створити клас Квадратне рівняння для квадратного рівняння --- наслідник першого класу,
з методом пошуку дійсних розв'язків.
Створити клас Бікваратне рівняння для біквадратного рівняння --- наслідник другого класу,
з методом пошуку дійсних розв'язків. В усіх класах передбачені методи введення/виведення та задання 
відповідно двох та трьох дійсних коефіцієнтів.
Введіть масив рівнянь з текстового файлу та знайдіть:
\begin{itemize}
\item
всі рівняння, що мають нескінчену кількість розв'язків;
\item
кількість рівнянь, що не мають дісних розвя'зків;
\item
найменший за модулем розв'язок;
\item
суму квадратів всіх дійсних розв'язків.
\end{itemize}

\item
  Опишіть клас Car, що має метод go(distance), який змінює пройдений
  кілометраж автомобілем та залишок пального. Метод go(\ldots{})
  залежить від віртуального методу fuelPerKm(), який визначає скільки
  потрібно пального автомобілю для проїзду одного кілометру. Нехай
  Personal (легковий автомобіль) і Truck (вантажівка) -- класи, що
  наслідують клас Car і перевизначають метод fuelPerKm(). При цьому
  потрібно врахувати, що цей метод залежить від кількості пасажирів
  (+10\% на кожного пасажира) для авто класу Personal або ваги вантажу
  для Truck (+25\% на кожну тонну вантажу). Визначити чи зможе задане
  авто проїхати задану відстань.

\item
Визначить клас Рівняння для однієї змінної. Клас дозволяє задавати інтервал,
де шукається корінь та має метод для знаходження кореня.
Створять наслідники цього класу: лінійне рівняння, кубічне рівняння, сінус,
експоненціальне рівяння, які дозволяють ввести параметри та коефецієнти таких типів
рівнянь. Реалізувати метод визначення коренів методом бієкція або іншими
в різних класах. Реалізуйте відповідні методи відбраження таких рівнянь.
Введіть масив рівнянь та:
\begin{itemize}
\item
виведіть всі рівняння, що не мають дійсних розв'язків;
\item
найбільший розв'язок;
\item
чи є інтервал, на якому у всіх рівнянь є хоча б один дійсний розв'язок;
\item
суму всіх дійсних розв'язків.
\end{itemize}

\item
Визначить базовий клас Товар (назва, артикул, одиниця виміру, вартість, дата поставки товару) та відповідні наслідники:
Іграшки(вікові обмеження), Їжа(час годності), Техніка(наявність гарантії, час гарантії).
Створіть бінарний файл з товарами та методи:
\begin{itemize}
\item
 пошуку даного товару(по назві та по типу) ---
виводити чи є даний товар, та якщо є - список всіх товарів що було знайдено; 
\item
оформлення заказу (вибір декількох товарів, підрахунок їх сумарної вартості та видалення заказаних товарів з файлу);
\item
зниження вартості товарів, час годності чи часу гарантії на них менше 2 днів на 20\%.
\end{itemize}

\item
Створіть клас Адреса, що містить рядкові поля Місто, Вулиця, та числові номер дома та квартири. 
Створять від нього наслідника Міжнародна адреса, що додає також до класу рядкові поля країна та почтовий код.
Введіть масив адрес та знайдіть найпопулярніше місто в даних адресах для якого також було введено як Міжнародна адреса. 
Запишить у текстовий файл всі адреси з цим містом доповниши всі адреси що були введені без міжнародних даних
за допомогою відомостей, що дало введення міжнародної адреси для цього міста.

\item
Створіть абстрактний клас Число з методами введення/виведення, додавання, множення, ділення.
Створіть класи Раціональне число та Комплексне число як наслідники цього класу. 
За допомогою даних класів створить функцію введення поліному від таких чисел
та обчисліть їх значення в даній Числовій точці.


\end{enumerate}

\end{document}
